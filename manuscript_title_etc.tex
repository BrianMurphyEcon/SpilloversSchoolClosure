%%%%%%%%%%%%%%%%%%%%%%%%%%%%%%%%%%%%%%%%
% 1. Define Keywords, JEL
%%%%%%%%%%%%%%%%%%%%%%%%%%%%%%%%%%%%%%%%
\newcommand{\PAPERKEYWORDS}{
school closure, spillovers}
\newcommand{\PAPERJEL}{\textbf{JEL}: J15, I24, O15}

%%%%%%%%%%%%%%%%%%%%%%%%%%%%%%%%%%%%%%%%
% 2. Define Title
%%%%%%%%%%%%%%%%%%%%%%%%%%%%%%%%%%%%%%%%
\newcommand{\PAPERTITLE}{Creative Title}

%%%%%%%%%%%%%%%%%%%%%%%%%%%%%%%%%%%%%%%%
% 3. Define Authors contact information
%%%%%%%%%%%%%%%%%%%%%%%%%%%%%%%%%%%%%%%%

\newcommand{\AUTHORHANNUM}{Emily Hannum}
\newcommand{\AUTHORHANNUMURL}{https://orcid.org/0000-0003-2011-9984}
\newcommand{\AUTHORHANNUMINFO}{\href{\AUTHORHANNUMURL}{\AUTHORHANNUM}: Department of Sociology and Population Studies Center, University of Pennsylvania, 3718 Locust Walk, Philadelphia, Pennsylvania, USA (\href{mailto:hannumem@sas.upenn.edu}{hannumem@sas.upenn.edu})}

\newcommand{\AUTHORWANG}{Fan Wang}
\newcommand{\AUTHORWANGURL}{https://orcid.org/0000-0003-2640-5420}
\newcommand{\AUTHORWANGINFO}{\href{\AUTHORWANGURL}{\AUTHORWANG}: Department of Economics, University of Houston, Houston, Texas, USA (\href{fwang26@uh.edu}{fwang26@uh.edu})}

\newcommand{\AUTHORKIM}{Jeonghyeok Kim}
\newcommand{\AUTHORKIMURL}{https://orcid.org/0009-0001-5127-1789}
\newcommand{\AUTHORKIMINFO}{\href{\AUTHORKIMURL}{\AUTHORKIM}: Department of Economics, University of Houston, Houston, Texas, USA (\href{jkim124@uh.edu}{jkim124@uh.edu})}

\newcommand{\AUTHORMUNSHI}{Rohit Munshi}
%\newcommand{\AUTHORKIMURL}{https://orcid.org/0009-0001-5127-1789}
\newcommand{\AUTHORMUNSHIINFO}{{\AUTHORMUNSHI}: Department of Economics, University of Houston, Houston, Texas, USA (\href{rmunshi@cougarnet.uh.edu}{rmunshi@cougarnet.uh.edu})}

\newcommand{\AUTHORMURPHY}{Brian Murphy}
%\newcommand{\AUTHORKIMURL}{https://orcid.org/0009-0001-5127-1789}
\newcommand{\AUTHORMURPHYINFO}{{\AUTHORMURPHY}: Department of Economics, University of Houston, Houston, Texas, USA (\href{bmmurphy2@uh.edu}{bmmurphy2@uh.edu})}


%%%%%%%%%%%%%%%%%%%%%%%%%%%%%%%%%%%%%%%%
% 4. Define Thanks
%%%%%%%%%%%%%%%%%%%%%%%%%%%%%%%%%%%%%%%%
\newcommand{\ACKNOWLEDGMENTS}{We are grateful to Aimee Chin and Chinhui Juhn for their invaluable guidance and support. For helpful comments, we thank Blake Heller, Fan Wang, Emily Hannum, Yona Rubinstein, Elaine Liu, Vikram Maheshri, Willa Friedman, Steven Craig, Saerom Ahn, Gergely Ujhelyi and participants at the University Houston Micro Workshop. We would also like to thank Jeanette Narvaez and Catherine Horn from the Education Research Center for their help while accessing the data. The research presented here uses confidential data from the State of Texas supplied by the Education Research Center (ERC) at The University of Houston (UH 080). All errors are our own. The conclusions of this research do not necessarily reflect the opinion or official position of Texas Education Research Center, the Texas Education Agency, the Texas Higher Education Coordinating Board, the Texas Workforce Commission, or the State of Texas.

}

%%%%%%%%%%%%%%%%%%%%%%%%%%%%%%%%%%%%%%%%
% 5. Define Abstract
%%%%%%%%%%%%%%%%%%%%%%%%%%%%%%%%%%%%%%%%
%\newcommand{\PAPERABSTRACT}{
%\noindent This paper shows that the neighborhoods where students grow up significantly influence their choice of college major. Using administrative data on Texas students who move across counties and school districts, we leverage variation in the timing of moves to estimate the causal effects of neighborhood exposure on the likelihood of choosing a STEM major. We find that students who relocate to high-STEM neighborhoods—defined by the share of non-moving peers obtaining STEM degrees—show a linear increase in their probability of majoring in STEM with each additional year spent in the area. We find cohort, gender, and major specific exposure effects, ensuring these effects are causal and not driven by selection or broader improvements in educational attainment. We also show that neighborhood STEM exposure closely tracks the local occupational composition, particularly the proportion of residents employed in STEM fields. Finally, we document that the positive impacts of living in STEM-rich neighborhoods extend broadly, benefiting students from disadvantaged backgrounds as well. Our findings underscore the role of neighborhood contexts in shaping educational pathways and highlight local environments as critical factors influencing educational inequality and STEM workforce development.

%}

\newcommand{\PAPERABSTRACTLONGER}{

}


%%%%%%%%%%%%%%%%%%%%%%%%%%%%%%%%%%%%%%%%
% 7. Define citation or availability of latest draft
%%%%%%%%%%%%%%%%%%%%%%%%%%%%%%%%%%%%%%%%
\newcommand{\PAPERDOIURL}{}
%\newcommand{\PAPERINFO}{This latest version of this paper is available at this 
%\href{https://jeonghyeok-kim.github.io/assets/GlobalChildrenTeachersSchools_HannumKimWang.pdf}{link}.
%}